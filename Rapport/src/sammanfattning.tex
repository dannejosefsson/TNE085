\begin{abstract}
Rapporten omfattar design och konstruktion av en radiostyrd svävare, där både
mekanisk och elektronisk hårdvara samt mjukvara har utvecklats av
projektgruppen. Fokus har legat på moduläritet för att få slutprodukten att fungera bra som en utvecklingsplattform. Projektgruppens förhoppning att detta projekt kan ligga
till grund för många framtida studentprojekt, från årskurs 1 fram till examen.

Svävarens elektroniska huvudsystem är baserade på olika Androidplattformar och
för styrning av egenutvecklad elektronikhårdvara så användes Arduino ADK, som är
en portning av Googles Android Development Kit (ADK).

Rapporten täcker hårdvaruutveckling av H-bryggor samt strömförsörjningkort med
med flera olika specifikationer. Altium Designer 6 har använts för utvecklingen
av kretskorten.

Mjukvaran på Androidplattformarna specifikt utvecklade för svävaren är skrivna i
Java och testade på ett flertal olika plattformar.

Kommunikation mellan plattformen ombord och fjärrkontrollen sker via
Bluetooth. Kommunikation mellan plattformen ombord och ADKn sker via USB.
Protokollet som används är baserat på Googles Protocol Buffer (PB) och
kommandona är egenutvecklade. Detta för att möjliggöra konvertering av implementationer av 
datatyper i olika språk. Nanopb, som är en portning av PB till inbyggda system,
används ombord på ADKn för att använda mindre av det begränsade minnet.

Styrsystemet är modulärt uppbyggt och baserar sig på designmönstrena Composite
och Builder. Detta gör det lätt att lägga till nya styralgoritmer. Det är även
förberett för ett reglersystem, men det hanns ej implementeras.

Den mekaniska designen har gjorts i SolidWorks 2012 och slutprodukten stämmer
bra överens med de tredimensionella ritningarna. Att noggranna ritningar gjordes
underlättade mycket vid konstruktion av luftkjolen.

Projektgruppen vill avsluta denna sammanfattning med att tacka Norrköpings
Polytekniska Förening för utdelat stipendium. Utan detta hade det inte varit
möjligt att genomföra projektet. Projektgruppen vill också tacka Björn-Åke Sköld
för hjälp och material vid konstruktionen av den mekaniska delen av projektet.
\end{abstract}
\thispagestyle{empty}
\pagebreak