\section{Projektdokument}
Under kursen har ett flertal dokument tagits fram. Dessa dokument sammanfattas nedan.
\subsection{Kravspecifikation}
I kravspecifikationen finns de krav på svävaren som togs fram i början av kursens gång. 
Detta dokument användes sedan som ett underlag för de beslut som togs under kursens gång.

I kravspecifikationen ställdes kravet att alla högsta prioritetskrav måste vara uppfyllda innan 
arbete på uppgifter med lägre prioritet fick påbörjas. Detta krav har uppfyllts och har hjälpt 
gruppen att behålla sitt fokus på de viktiga uppgifterna även när det ibland har varit lockande 
att jobba med någon roligare eller mer intressant uppgift.

Flertalet av prio 1 kraven uppfylldes, några undantag fanns då vissa krav visade sig vara överflödiga.
Ett exempel på ett sådant krav är krav 54 (I kravspecifikation version 0.6) ``Vid 3.3 V får spänningen
avvika med max 0.2V.'' som inte uppfylls då ingen elektronik på svävaren behöver 3.3 V och det därmed inte 
finns något spänningskort som levererar 3.3 V.

Kravspecifikationen finns i appendix [XXXXXXXXXX].
\subsection{Projektplan}
I projektplanen sammanfattas den planering av arbetet som utfördes tidigt i projektet.

Projektplanen finns i appendix [XXXXXXXXXX].
\subsection{Kodstandard}
En kodstandard skrevs och har i stor utsträckning följts under projektets gång.

Kodstandarden finns i appendix [XXXXXXXXXX].