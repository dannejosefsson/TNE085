\subsection{Loggning av data}
Börja med att trycka på knappen "Settings” på huvudskärmen i figur
\ref{fig:remote} . Då kommer inställningsmenyn upp och loggning av data kan
konfigureras. Det går inte att trycka på ”Start Log”  och starta loggen om inte
en eller flera sensorer valts under ”Settings”.
Observera att loggning enbart kan ske när blåtandskommunikationen är igång.

Under inställningsmenyn i figur \ref{fig:remoteSettings} finns tre sensorer att
välja på. Genom att kryssa i rutan framför aktuell sensor får programmet order
om att logga och spara datan för den aktuella sensorn på det externa
minneskortet på fjärrkontrollen. För att gå tillbaka till huvudskärmen tryck på
”Back to Main Menu” och starta loggen genom att trycka ”Start Log” som ses i
figur \ref{fig:remote} . För att stoppa loggen tryck ”Stop Log”. Om användaren
går in under ”Settings” och avmarkerar en sensor under pågående loggning så
avbryts loggningen omedelbart.