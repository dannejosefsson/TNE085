\section{Bakgrund}
Inom ramarna för projektkursen TNE085 på Tekniska högskolan vid Linköpings
universitet så skapades en svävare. Valet att bygga en svävare gjordes för att
projektet skulle täcka in så många delar av gruppmedlemmarnas utbildning som
möjligt.

Vidare så fanns det sedan ett tidigare projektarbete på skolan; en autonom
svävare, som verkade som en stor inspirationskälla inför arbetet. En svaghet i
det projektet är att plattformen är relativit låst för vidare utveckling. Detta beroende på
hur komponenter är designade och plattformen i sig är konstruerad. Huvudmålet
i detta projekt är att svävaren ska bli en bra plattform för kommande
utvecklingsprojekt vid universitetet. Där flera möjligheter till
vidareutveckling skall finnas tack vare gedigen och modulär design av svävaren.
