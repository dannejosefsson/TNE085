\section{Bakgrund}
Inom ramarna för projektkursen TNE085 på Tekniska högskolan vid Linköpings
universitet så skapades en svävare. Valet att bygga en svävare gjordes för att
projektet skulle täcka in så många delar av gruppmedlemmarnas utbildning som
möjligt.

Vidare så fanns det sedan ett tidigare projektarbete på skolan; en autonom
svävare, som verkade som en stor inspirationskälla inför arbetet. En svaghet i
det projektet är att plattformen är relativit låst för vidare utveckling. Detta beroende på
hur komponenter är designade och plattformen i sig är konstruerad. Huvudmålet
i detta projekt är att svävaren ska bli en bra plattform för kommande
utvecklingsprojekt vid universitetet. Där flera möjligheter till
vidareutveckling skall finnas tack vare gedigen och modulär design av svävaren.

\section{Inledning}
I rapporten förklaras det vilka projektdokument som har producerats under
projektets gång vilka parter som ingick. Under ingående system så redovisas
vilken hårdvara som har inhandlats respektive designats av projektgruppen.
Mjukvaruavsnittet behandlar de olika programmen som har utvecklats, samt vilken
extern mjukvara som har använts. Då både utvecklingsmiljöer samt externa
ramverk och verktyg som har använts tillsammans med den egenutvecklade. Det
rapporteras också kring hur mycket pengar som har spenderats under projektets
gång.
