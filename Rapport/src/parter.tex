\section{Parter}
Fyra olika parter var delaktiga under projektet; projektgruppen,
beställargruppen, Björn-Åke Sköld och Norrköpings Polytekniska Förening. De
olika parternas medverkan presenteras i följande underavdelningar.

\subsection{Projektgrupp}
Projektgruppens medlemmar, kontaktuppgifter och respektive ansvarsområden
redovisas i Tabell \ref{tbl:projektgruppen}

\begin{table}[htbp!]
\label{tbl:projektgruppen}
\caption{Projektgruppen och medlemmarnas ansvarsområden.}
\centering
\begin{tabular}{|l|l|p{0.50\textwidth}|}
\hline
Namn & LiU-ID & Ansvarsområden \\
\hline
\hline
Daniel Josefsson & danjo140 & Projektledare, styr- och reglersystem \\
Viktor Johansson & vikjo493 & Reglersystem \\
Rickard Dahm & ricda841 & ADK (C/C++), dokument \\
Olle Kalered & ollka965 & Sekreterare, test och kvalitet, fika \\
Jens Moser & jenmo917 & Android (Java), ADK (C/C++) \\
Emil Andersson & emian195 & Inköp och ekonomi \\
Johan Gustafsson & johgu962 & Android (Java) \\
Jesper Cronborn & jescr691 & Hårdvara (mekanik och konstruktion)\\
Rikard Israelsson & rikis126 & Tids- och lokalallokering, \newline 
Hårdvara (CAD) \\
\hline
\end{tabular}
\end{table}

Medlemmarna har inte enbart arbetat med de områdena som de ansvar över, utan
arbetet har delats upp efter kunskap, intresse och förmåga över områdena.

\subsection{Beställargrupp}
Beställargruppen bestod av Ole Pedersen och Gustav Knutsson. Dessa hade i
uppdrag att skärskåda de dokument som projektgruppen producerade, följa upp
arbetet samt att bidra med teknisk kompetens.

\subsection{Björn-Åke Sköld}
Björn-Åke Sköld är till vardags industridesigner och ägare av Sköld Design AB.
Han verkade som sakkunning hjälp till projektgruppen vid design och konstruktion
av svävarplattformen. Björn-Åke har konstruerat flertalet svävare, varav flera
stora nog att bära passagerare.

\subsection{Norrköpings Polytekniska Förening}
Norrköpings Polytekniska Förening, NPF, äger en stipendiefond och valde att dela
ut ett stipendium på 5000 kr till bygget av svävaren. Utan det stipendiet hade
inte byggnationen varit möjlig.
