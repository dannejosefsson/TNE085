\subsection{Batterier}
En studie av olika batteritekniker gjordes för att se vilka som skulle klara
kraven. Valet föll på LiPo-tekniken, då denna har hög effektdensitet och är
kostnadseffektiv.

\subsubsection{Resultat}
Två stycken 5 cells LiPo-batterier på 18,5 V och 3.3 Ah används, där ett går
till enbart strömförsörjningen för lyftfläktarna (då dessa kräver högst spänning
och mest ström) och det andra till resterande elektronik. Batteriernas
specifikation kan läsas i Tabell \ref{tbl:Battery}.

\begin{table}[htbp!]
\centering
\caption{Batterispecifikation}
\label{tbl:Battery}
\begin{tabular}{l|l}
Egenskap & Värde \\
\hline
Minsta strömkapacitet & 3300 mAh\\
Konfiguration & 5S1P / 18.5V / 5Cell\\
Maximal konstant urladdning & 30C\\
Maximal toppurladdning (10 s) & 40C\\
Vikt & 480g\\
Dimensioner (l * b * h) & 139 mm * 43 mm * 35mm\\
\end{tabular}
\end{table}

Spänningsnivån på 18,5 V är framförallt på grund av att regulatorn på
strömförsörjningskortet behöver ha minst 15 V in för att kunna hålla 12 V eller
högre ut. 3.3 Ah för att klara kraven på 10 minuters drifttid vid max
påfrestning. Inget test av maximal drifttid har gjorts men tester med minst 30
minuters körning klarades.

\subsubsection{Diskussion}
Batterierna mötte kraven väl. Eventuellt så skulle man kunna minska
strömkapaciteten för att minska totalvikten på svävaren. Projektgruppen tyckte
inte att den extra kostnaden, både i tid och pengar, samt
att försämringen av prestandan var värd inköp av andra batterier.

\subsubsection{Ekonomi}
Inköpet av batterier kostade 979 SEK, inklusive tull och frakt.
