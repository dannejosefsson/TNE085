\subsection{Strömförsörjning}
Strömförsörjningen designades med moduläritet i  åtanke så att flera olika
spänningsnivåer uppnås genom samma design på kretskort. De olika
strömförsörjningskorten som det designades för är:
\begin{itemize}
	\item 3,3 V, 3 A, med en felmarginal på $\pm$0.2 V.
	\item 5 V, 3 A, med en felmarginal på $\pm$0.5 V.
	\item 12 V, 3 A, med en felmarginal på $\pm$0.8 V.
	\item En justerbar spänningsnivå med en felmarginal på $\pm$10 \% av den
önskade spänningen och kunna leverera 3 A.
\end{itemize}

Varje typ av strömförsörjning skulle finnas på ett separat PCB, dock med samma
layout för enkel tillverkning och att eventuell beställning av kort skulle vara
billig.

Strömförsörjningarna designades också så att de gick att parallellkoppla, både
på ingångs- och utgångssidan. Detta för att kunna utöka kapaciteten vid behov.

För att uppnå kraven och för att nå god effektivitet så används switchade
regulatorer, för mer teori och val av design och komponenter se Appendix
\ref{apx:PSU}.

\subsubsection{Resultat}

Totalt 6 stycken strömförsörjningskort används i svävaren och versionerna som
konstruerades är:
\begin{itemize}
\item 2 stycken justerbara 13.8 V som driver lyftfläktarna (ett
strömförsörjningskort till två fläktar).
\item 2 stycken justerbara 10 V för drivning av drivfläktarna (ett
strömförsörjningskort per fläkt).
\item 1 stycke 12 V till drivkretsarna på båda H-bryggorna.
\item 1 stycke 5 V för ADK och logikkretsarna på båda H-bryggorna.
\end{itemize}

Korten klarar av att hålla både ström och spänningsnivåerna bra utan störningar
eller spikar.

\subsubsection{Diskussion}
Då kraven på strömförsörjningen specificerades så var det inte bestämt om det
skulle byggas ett nytt chassi med andra motorer eller inte. Därför så valdes
spännings och strömnivåer från den gamla svävaren som grund och korten
designades därefter. Detta ledde till att det behövdes fler
strömförsörjningskort än planerat. Men det medförde dock inte några problem och
detta tack vare det modulära tänkandet.

Vid för tung belastning, t.ex med starkare motorer och fläktar,
sjunker spänningen och strömmen ökar och därav bör det användas säkringar
på 4-6~A.

Det visade sig att de justerbara spänningsnivå korten var ett mycket bra val, då
vi lätt kunde öka nivån för fläktar och motorer så bäst funktion kunde erhållas.
Vid de högre spänningarna är det värt att notera att det behövdes kylflänsar
på regulatorerna.

Strömförsörjnningskorten har plats för filterkomponenter både innan regulatorn
och efter, dessa platser är i nuläget tomma och förbikopplade då inga större
störningar eller spikar kunnat uppmätas.

\subsubsection{Förbättringar}
I själva designen för strömförsörjningskorten så bör man inkludera en en
diod som förhindrar backspänning från de andra korten som lägger sig
över avstängda kort. Problemet kan återskapas lätt genom att stänga av ett
strömförsörjningskort men att ha det sammankopplat med andra
strömförsörjningskort som är i drift. Då kan statusdioder lysa trots att kortet
är avstängt. Den backspänning som uppkommer gör ingen skada på
strömförsörjningskorten, men det är lite förvillande att ett korts statusdioder
lyser trots att kortet är avstängt. Notera även  att det är H-bryggorna som
bidrar med att denna backspänning uppstår då de har flera spänningar inkopplade.

Då nuvarande strömförsörjning inte klarar att ge mer än 3 A vid sin
spänningsnivå begränsar det, mer än förväntat, val av motorer och fläktar. Så
ett strömförsörjningskort designat för högre ström skulle vara en klar
förbättring. Detta skulle också minska antalet strömförsörjningskort.
För detta kort rekommenderas det att man ej använder en färdig regulator utan
själv bygger upp det för att få bra funktion och strömbegrännsning.

\newpage
\subsubsection{Ekonomi}
Kostnad för komponenterna för de olika spänningsnivåerna ses i tabellerna \ref{gemensammadelar} - \ref{justerbar}.

\begin{table}[htbp]
\centering
\caption{Gemensamma delar}
\begin{tabular}{|l|l|r|r|}
\hline
\textbf{Komponent} & \textbf{Information} & \textbf{Antal} & \textbf{Pris} \\ 
\hline
Kondensator & 680~$\mu$F, 35V & 1 & 5,68 \\ 
\hline
Schottky diod & SB540 & 1 & 6,05 \\ 
\hline
Tryckvippströmställare & R19U-R112A-B-2-B-M2 & 1 & 9,03 \\ 
\hline
Molex hane & 2p & 2 & 4,56 \\ 
\hline
Stiftdon 90\degree & 2p & 2 & 9,86 \\ 
\hline
Säkringshållare & 6.7~A & 2 & 3,07 \\ 
\hline
Lysdiod & 1224-10SYGC/S530-E2 & 2 & 1,49 \\ 
\hline
\textbf{Total} &  & \multicolumn{1}{l|}{} & 39,74 \\ \hline
\end{tabular}
\label{gemensammadelar}
\end{table}


\begin{table}[htbp]
\centering
\caption{5~V Strömförsörjning}
\begin{tabular}{|l|l|r|r|}
\hline
\textbf{Komponent} & \textbf{Information} & \textbf{Antal} & \textbf{Pris} \\
\hline
Switch regulator, 5V, TO-220  & LM2596T33 & 1 & 52,62 \\ 
\hline
Spole &  47~$\mu$H  & 1 & 16,73 \\ 
\hline
Kondensator & 330~$\mu$F, 35V & 1 & 5,86 \\ 
\hline
\textbf{Total} &  & \multicolumn{1}{l|}{} & 114,95 \\ 
\hline
\end{tabular}
\label{5vstrom}
\end{table}


\begin{table}[htbp]
\centering
\caption{12~V Strömförsörjning}
\begin{tabular}{|l|l|r|r|}
\hline
\textbf{Komponent} & \textbf{Information} & \textbf{Antal} & \textbf{Pris}  \\ 
\hline
Switch regulator, 12V, TO-220  & LM2596T-12  & 1 & 59,58 \\ 
\hline
Spole & 47~$\mu$H  & 1 & 16,73 \\ 
\hline
Kondensator & 180~$\mu$F, 35V & 1 & 2,33 \\ 
\hline
\textbf{Total} &  & \multicolumn{1}{l|}{} & 118,38 \\ 
\hline
\end{tabular}
\label{12vstrom}
\end{table}


\begin{table}[htbp!]
\centering
\caption{Justerbar spänning strömförsörjning}
\begin{tabular}{|l|l|r|r|}
\hline
\textbf{Komponent} & \textbf{Information} & \textbf{Antal} & \textbf{Pris} \\
\hline
Switch regulator, justerbar, TO-220  & LM2596T-ADJ  & 1 & 56,73 \\ 
\hline
Spole  & 68~$\mu$H & 1 & 16,73 \\ 
\hline
Kondensator & 220~$\mu$F, 35V  & 1 & 3,91 \\ 
\hline
\textbf{Total} &  & \multicolumn{1}{l|}{} & 117,11 \\ 
\hline
\end{tabular}
\label{justerbar}
\end{table}

Även dessa kort beställdes från ITead Studio för en kostnad av 369,50 SEK, vilket ger en sammanlagd summa av 759,68 SEK för alla strömförsörjningskort.



