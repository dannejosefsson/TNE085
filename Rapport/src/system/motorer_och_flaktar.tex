\subsection{Motorer och fläktar}
De lyftfläktar som används är fyra stycken centrifugalfläktar som generellt sett
inte ger lika mycket luftflöde som axiella fläktar men ger däremot mycket högre
lufttryck istället. Genom att placera de fyra fläktarna parallellt med varandra
blir luftflödet ungefär fyra gånger större än för en fläkt och ger därmed ett
tillräckligt luftflöde. Ett högt lufttryck ger bidrag till att svävaren kan bära
en större totalvikt. Se specifikation av centrifugalfläkten i Tabell
\ref{tbl:fan_spec} och se mer information i datablad \cite{Delta_BFB1212VH-R00}
. Mer information kring hur lyftkraftsberäkningarna gjordes fås i Appendix
\ref{app:lyftkraftsberakningar}.

\begin{table}[htbp!]
\centering
\caption{Specifikation av centrifugalfläkt.}
\label{tbl:fan_spec}
\begin{tabular}{|l|r|}
\hline
Märkspänning & 12 V\\
\hline
Driftspänning & 4,0 - 13,8 V\\
\hline
Ström & 1,25 A\\
\hline
Effekt & 15,00 W\\
\hline
Rotationshastighet & 3100 rpm\\
\hline
Maximalt luftflöde & 1,120 m$^3$/min\\
\hline
Maximalt lufttryck & 323,6 Pa (33,00 mmH$_2$O)\\
\hline
Ljudnivå & 56,5 dB-A\\
\hline
\end{tabular}
\end{table}

Till drivningen används två stycken elmotorer (DC) som driver varsin propeller.
Se specifikation av motorn i Tabell \ref{tbl:motor_spec} och se mer information
i datablad, \cite{Motraxx_XFLY400-12}.

\begin{table}[htbp!]
\centering
\caption{Specifikation av elmotor.}
\label{tbl:motor_spec}
\begin{tabular}{|l|r|}
\hline
Märkspänning & 12 V\\
\hline
Driftspänning & 6 - 15 V\\
\hline
Tomgångsvarvtal & 21000 rpm\\
\hline
Tomgångsström & 0,6 A\\
\hline
Maximal effekt & 42,1 W\\
\hline
VID MAXIMAL EFFEKTIVITET &\\
\hline
Lastvarvtal & 19000 rpm\\
\hline
Strömförbrukning & 2,5 A\\
\hline
Vridmoment & 1,00 Nmm\\
\hline
Avgiven effekt & 19,5 W\\
\hline
Effektivitet (verkningsgrad) & 65\%\\
\hline
\end{tabular}	
\end{table}
