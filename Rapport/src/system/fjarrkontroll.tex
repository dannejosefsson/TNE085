\subsection{Fjärrkontroll}
Fjärrkontrollens huvudsakliga uppgift är att styra svävaren. Den fysiska fjärrkontrollen utgörs av en generell Androidplattform med Android API 10 eller högre (hittills har tester dock bara genomförts med plattformer av smartphone-modell). På denna plattform körs sedan en Androidapplikation skriven i Java. 

Applikationen är uppdelad tre huvudsakliga delar som var och en utgörs av en Android-service. Dessa delar implementerar funktionalitet för kommunikation, loggning av sensordata samt generering av styrsignaler. Applikationen innehåller även Android-activity som fungerar som inställningsmeny för fjärrkontrollen.

\subsubsection{Kommunikation}
Kommunikationen mellan svävaren och fjärrkontrollen sker via en Bluetoothlänk implementerad med hjälp av Android Bluetooth API, där svävaren agerar som Bluetooth Server. I rollen som Bluetooth Server använder sig svävaren av Bluetoothprofilen Serial Port Profile (SPP), som i stora drag upprättar en trådlös seriell kommunikationslänk som fjärrkontrollen sedan kan ansluta sig till. 
Fjärrkontrollen har funktionalitet för att söka upp och välja vilken Bluetooth-enhet den ska anslutas till, vilket gör att det är lätt att ansluta till svävaren även om det finns andra aktiva enheter i omgivningen.
För den mesta dataöverföringen mellan svävaren och fjärrkontrollen används Protocol Buffer som finns beskrivet i […].

\subsubsection{Loggning av sensordata}
På fjärrkontrollen finns även funktionalitet för att logga olika typer av sensordata. Insamlade data sparas som textfiler på fjärrkontrollens SD-minne. 

På den senaste versionen av applikationen finns funktionalitet för loggning av data från fjärrkontrollens accelerometer, svävarens accelerometer samt från svävarens ultraljudssensorer. Vilken data som skall loggas ställs in i inställningsmenyn. Om data från svävaren önskas skickas ett kommando till den som gör att den svarar med ett paket innehållande önskad data. 

\subsubsection{Generering av styrsignaler}
På fjärrkontrollen genereras även de styrsignaler som används för att styra svävaren. Detta behandlas mer i […]

\subsection{Ekonomi}
Ur en ekonomisk aspekt är det väldigt trevligt att jobba med utveckling av Androidapplikationer, både utvecklingsvertyget Android SDK och utvecklingsmiljön Eclipse är gratis.

\subsection{Resultat}
Fjärrkontrollen möter de krav som ställs i kravspecifikationen.

\begin {itemize}
\item Räckvidd 20 meter
\item Indikera när kommunikationen med svävaren bryts.
\item Kunna kommunicera med svävaren. Samt ge de kommandon som gränssnittet till svävaren tillåter.
\end {itemize}

\subsection{Diskussion}
Det finns många fördelar med att använda en Android plattform som fjärrkontroll. Många av dessa plattformer, om inte alla, är utrustade med Bluetooth som är lätt att implementera i egna applikationer i och med Android Bluetooth API. 

Dessa plattformar har även en hel uppsättning av olika sensorer, så som accelerometer. Accelerometern går t.ex. att användas till att generera styrsignaler genom att vinkla plattformen.

Det är även trevligt att kunna använda sig av olika enheter som fjärrkontroll ur ett demoperspektiv, då vem som helst med t.ex. en egen Androidtelefon kan ladda hem applikationen och ansluta sig som fjärrkontroll. 






