\subsection{Fjärrkontroll}
\label{subsec:system/fjarrkontroll}
Fjärrkontrollens huvudsakliga uppgift är att styra svävaren. Den fysiska
fjärrkontrollen utgörs av en generell Androidplattform med Android API 10 eller
högre. Fjärrkontrollen ska ha tillräcklig hårdvara för att kunna skapa
och sända styrsignaler till Routern.

En stor del av Sveriges befolkning äger också en Androidplattform och detta gör
att man kan bruka dessa som fjärrkontroller vid demonstrationer eller dylikt, om
man installerar behövlig programvara. Mer om fördelar vid mjukvaruutveckling kring
Android och projektgruppens implementationer finns i avdelning
\ref{sec:Funktionalitet}, Mjukvara.

En annan av fjärrkontrollens uppgifter är att logga data från sensorer och andra
händelser på svävaren. Datan ska till exempel kunna användas vid verifiering av
reglersystem och återskapande av genomfört rörelsemönster.

\subsubsection{Resultat}
Ingen specifik hårdvara inhandlades för att agera fjärrkontroll. Istället
användes projektgruppens privata plattformar vid utvecklingen.

\subsubsection{Ekonomi}
Då ingen specifik hårdvara inhandlades medförde det ej heller någon kostnad.

\subsubsection{Diskussion}
Det finns många fördelar med att använda en Androidplattform som fjärrkontroll.
Många av dessa plattformar, om inte alla, är utrustade med Bluetooth och
WiFi. Dessa standardiserade kommunikationskanaler är väldigt användbara,
detta eftersom att man inte låser sig vid kommunikation mot en annan
Androidplattform.

Androidplattformar har även en hel uppsättning av olika sensorer, så som
accelerometer och tryckkänslig skärm. För att generera styrsignaler kan man
använda t.ex. accelerometervärdena som uppstår när användaren vinklar
plattformen eller den tryckkänsliga skärmen som en mer traditionell fjärrkontroll.

Det är även trevligt att kunna använda sig av olika enheter som fjärrkontroll ur
ett demoperspektiv, då vem som helst med t.ex. en egen Androidplattform kan
ladda hem applikationen och ansluta sig som fjärrkontroll.
