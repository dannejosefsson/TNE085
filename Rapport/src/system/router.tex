\subsection{Router}
Routern fungerar som en kommunikationslänk mellan fjärrkontroll och ADK. 
När den tar emot meddelande kontrollerar den mottagare och skickar vidare till rätt destination. 
Programmet är en Android-app och skriven i Java.

Styrsystemet är för stunden implementerat på fjärrkontrollen men visionen är att ett fullfjädrat reglersystem skall 
köras på router-telefonen och därför har routern utformats på ett sådant sätt att det skall vara enkelt att 
implementera ett på sidan om routerfunktionen.

Programmet är indelat i tre Android-services. En Blåtandsservice som hanterar all blåtandskommunkation. 
En UsbService som hanterar all USBkommunkation och slutligen en Service för ett framtida reglersystem.
Kommunkationen inom programmet sker via Android Broadcasts.

Applikationen startas automatiskt genom att ansluta ADK:n till router-telefonen. 
Om applikationen körs innan den ansluts till ADK:n kommer den att startas om för att upprätta USB-kommunikationen. 
När appen körs visas en diod för respektive kommunkationsbuss. Om kommunkationen fungerar lyser dioden grönt och om 
den inte fungerar så lyser den rött. Dioden kan även lysa gult men detta är inget man hinner se i vanliga fall då 
den enbart gör så under uppkoppling. Knappen "Bluetooth setup" används för att upprätta en blåtandslänk.

\subsubsection{Ekonomi}
Det behövs en Android-telefon för att köra programmet och vi använde en Samsung Google Nexus S för 1400SEK. 
Anledningen till att vi valde denna telefon var för att den fungerar bra ihop med ADK:n, den har en öppen 
mjukvara vilket gjorde lätt att ladda in de senaste uppdateringarna men även för att den var ett av de billigare 
alternativen.

\subsubsection{Resultat}
Programmet är stabilt och uppfyller alla krav. Dess huvudsyfte är att ta emot och skicka vidare meddelanden 
vilket fungerar utmärkt.

\subsubsection{Diskussion}