\subsection{ADK}
Ett Android Development Kit (ADK) användes för att kontrollera svävaren. 
Ett ADK är ett utvecklingskort speciellt anpassat för att användas tillsammans med Android.
Det fanns några alternativ men gruppen valde att införskaffa två olika kort för att testa 
vilket som till slut skulle användas på svävaren. De två korten var Embedded Artists kort 
``Android Open Accessory Application Kit'' (AOAA) \cite{AOAA} och Arduinos ``Arduino ADK'' \cite{Arduino ADK}. 

\subsubsection{Ekonomi}
Då gruppen valde att införskaffa två olika utvecklingskort gick kostnaden upp något. Arduino ADK kostade 57.71 \euro och 
Embedded Artists AOAA kostade 65.18 \euro som vid införskaffandets växlingskurs uppgår till totalt 1385 sek. 

\subsubsection{Resultat}
Det bestämdes att kortet från Arduino skulle användas då kortet visade sig vara 
lättare att programera för än Embedded Artists men uppfyllde fortfarande de krav vi hade för svävaren. 

ADKn är kopplad till telefonen som sitter på svävaren (Router telefonen) via USB.
\subsubsection{Diskussion}
Det skulle gå att byta utvecklingskort helt och köra på Embedded Artists AOAA vilket öppnar upp för fler funktioner, 
t.ex kommunikation via CAN-bussen. Kortet är dock svårare att programmera för vilket är anledningen till att det inte användes
i detta projekt.