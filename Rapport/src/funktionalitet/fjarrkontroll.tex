 På denna plattform körs sedan
en Androidapplikation skriven i Java.

 Vid skrivandet av rapporten hade tester dock bara
genomförts på plattformar av smartphone-modell.

Applikationen är uppdelad tre huvudsakliga delar som var och en utgörs av en Android-service. Dessa delar implementerar funktionalitet för kommunikation, loggning av sensordata samt generering av styrsignaler. Applikationen innehåller även Android-activity som fungerar som inställningsmeny för fjärrkontrollen.

\subsubsection{Kommunikation}
En Bluetoothlänk sköter kommunikationen mellan svävaren och fjärrkontrollen.
Denna är implementerad med hjälp av Android Bluetooth API, där svävaren agerar
som Bluetooth Server. I rollen som Bluetooth Server använder sig svävaren av
Bluetoothprofilen Serial Port Profile (SPP), som i stora drag upprättar en
trådlös seriell kommunikationslänk som fjärrkontrollen sedan kan ansluta sig
till.
Fjärrkontrollen har funktionalitet för att söka upp och välja vilken
Bluetooth-enhet den ska anslutas till, vilket gör att det är lätt att ansluta
till svävaren även om det finns andra aktiva enheter i omgivningen.
För den mesta dataöverföringen mellan svävaren och fjärrkontrollen används
Protocol Buffer som finns beskrivet i kapitel \ref{subsubsec: Protocol buffer}.

\subsubsection{Loggning av sensordata}
På fjärrkontrollen finns även funktionalitet för att logga olika typer av sensordata. Insamlade data sparas som textfiler på fjärrkontrollens SD-minne. 
På den senaste versionen av applikationen finns funktionalitet för loggning av data från fjärrkontrollens accelerometer, svävarens accelerometer samt från svävarens ultraljudssensorer. Vilken data som skall loggas ställs in i inställningsmenyn. Om data från svävaren önskas skickas ett kommando till den som gör att den svarar med ett paket innehållande önskad data. 

\subsubsection{Generering av styrsignaler}
På fjärrkontrollen genereras även de styrsignaler som används för att styra
svävaren. Detta behandlas mer i kapitel \ref{subsec:styr och regler}.

Ur en ekonomisk aspekt är det väldigt trevligt att jobba med utveckling av
Androidapplikationer, både utvecklingsvertyget Android SDK och utvecklingsmiljön Eclipse är gratis.

\subsubsection{Resultat}
Fjärrkontrollen möter de krav som ställs i kravspecifikationen.


\begin {itemize}
\item Räckvidd 20 meter
\item Indikera när kommunikationen med svävaren bryts.
\item Kunna kommunicera med svävaren. Samt ge de kommandon som gränssnittet till svävaren tillåter.
\end {itemize}

som är lätt att implementera i egna applikationer i och med
Android Bluetooth API.