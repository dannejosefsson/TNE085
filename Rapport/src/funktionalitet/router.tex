\subsection{Router}
Routerns uppgifter är att handha reglersystemet och att agera som
kommunikationslänk mellan de olika enheterna.

Programmet är indelat i tre Android-services och en Android-activity:
\begin{itemize}
	\item Bluetoothservice som hanterar all Bluetoothkommunkation,
	\item UsbService som hanterar all USBkommunkation,
	\item ControlSystemService som implementerar reglersystemet
	\item och MainActivity som implementerar användargränssnittet.
\end{itemize}

Applikationen startas automatiskt genom att ansluta ADK:n till router-telefonen. 
Om applikationen körs innan den ansluts till ADK:n kommer den att startas om för
att upprätta USB-kommunikationen.
När appen körs visas en diod för respektive kommunkationsbuss. Om kommunkationen
fungerar lyser dioden grönt och om den inte fungerar så lyser den rött. Dioden
kan även lysa gult men detta är inget man hinner se i vanliga fall då den enbart
gör så under uppkoppling. Knappen "Bluetooth setup" används för att upprätta en
Bluetoothlänk.

Intern kommunkation inom applikationen sker via Android Broadcasts.
Kommunikationen med andra enheter behandlas i avsnitt \ref{subsec:commlink} och
reglersystemet behandlas i \ref{subsec:styr och regler}.

\subsubsection{Resultat}
Programmet är stabilt och uppfyller alla krav. Dess huvudsyfte är att ta emot
och skicka vidare meddelanden vilket fungerar utmärkt.