\subsection{Kommunikationslank}
Mjukvarugänget\ldots
\subsubsection{Protocol buffer}
Protocol Buffer\cite{Protocol buffer} är ett protokoll som används för
serialisering av data.
Protokollet är utvecklat av Google, där Google har gjort kompilatorer för Java,
C++ och Python tillgängliga. Utöver dessa kompilatorer från Google så finns det
andra kompilatorer utvecklade av privatpersoner eller företag för andra språk.

Svävaren använder sig dels av Googles javakompilator för protokollet i 
fjärrkontrollen samt telefonen. För ADK:n så används en simplare version av
Protocol Buffer nämligen Nanopb \cite{Nanopb} som implementerar protokollet i
statisk C kod.

Det Protocol Buffer generar är programkod med funktionalitet för att kunna
serialisera och deserialisera olika dataobjekt. Dessa dataobjekt beskrivs av
användaren med ett IDL (Interface Definition Language).
Svävaren använder sig av sju stycken olika dataobjekt för sin kommunikation och
dessa är definierade i mappen PB i källkoden \cite{Source code}. Filen
command\_1.proto är filen som definierar kommandon för Nanopb och filen
Command.proto definierar samma kommandon för javaprojekten.

\subsubsection{ADK till Telefon}
Rickard och Jens\ldots
\subsubsection{Telefon till fjärrkontroll}
Johan\ldots
